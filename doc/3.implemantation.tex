\section{Implementacja}

\subsection{Wieloplatformowość}

Przeglądarka jest napisana w taki sposób, że jej implementacja nie uwzględnia systemu operacyjnego na którym pracuje

Zróżnych perspektyw

\subsection{Język programowania}

Przeglądarka została napisana w C++ w standardzie z 2017 roku w skrócie C++17

\subsection{Środowisko programistyczne}

Do programowania głównie używałem CLion, IDE stworzonego przez firmę JetBrians.
Zdecydowana większość czasu przeglądarka była testowana i debugowana na aktualizowanym systemie ArchLinux.

\subsection{Obiektowy model w oprogramowaniu}

Cały projekt jest zaprojektowany w sposób obiektowy, za wyjątkiem kilku pomniejszych elementów.

\subsubsection{DicomScene}

Klasa dziedzicząca pośrednio po \qtclass{QGraphicsScene} przez \sokarclass{Scene}.
Jest to obiektem jednej ramki obrazu i jest odpowiedzialna za pośrednie wygenerowanie obrazu oraz jego wyświetlenie na ekranie.

Informacje na obrazie są wyświetlane za pomocą obiektów \sokarclass{SceneIndicator}.
Obiekty te mają dostęp do bazy danych obrazu DICOM i odpowiednio zmieniają swoją zawartość.

\begin{itemize}

    \item \sokarclass{PatientDataIndicator}

    Obiekt wyświetlający dane pacjenta, pojawia się od zawsze na obrazie i zawiera następujące linie:

    \begin{itemize}
        \item Nazwa pacjenta oraz płeć

        Nazwa pacjenta znajduje się w \dicomtag{PatientName}{0010}{0010}
        Jako, że pacjenta, bądź obiekt badany można nazwać w sposób dowolny i odbiegający od polskiego standardu nazewnictwa, standard DICOM nie przewiduje rozdzielenie poszczególnych składowych nazwy na oznaczone fragmenty.
        Do nazwy osoby został utworzony specjalny VR, \dicomvr{Person Name}{PN}, który dzieli nazwę na podane fragmenty, rozdzielony znakiem \texttt{\^{}} (94 znak kodu ASCII):
        \begin{itemize}
            \item family name complex - nazwisko, np. Smolik
            \item given name complex - imię, np. Adam
            \item middle name - środkowe imię, brak odpowiednika w polskim nazewnictwie
            \item name prefix - prefiks przed imieniem, np: mgr. inż.
            \item name suffix - sufiks po imieniu, brak odpowiednika
        \end{itemize}
        W przypadku mniejszej ilości segmentów, mamy założyć, że są puste
        Na przykład "prof. dr. hab. inż. Waldemar Smolik pracownik ZEJIM" był by zapisany w sposób następujący: \texttt{Smolik\^{}Waldemar\^{}\^{}prof. dr. hab. inż.\^{}pracownik ZEJIM}

        Płeć, zapisana jest w \dicomtag{PatientSex}{0010}{0040} i przedstawiana jest za pomocą znaków UTF-8, "O" i "O", odpowiednio mężczyzny i kobiety.
        W przypadku określenia inne płci niż jest w standardzie bądź braku tagu płeć nie będzie widoczna.

        \item Identyfikator pacjenta

        ID pacjenta z tagu \dicomtag{PatientID}{0010}{0020} wyświetlane w takiej formie jakiej jest zapisane, bez żadnej obróbki.

        \item Data urodzenia oraz wiek pacjenta

        Data urodzenia znajdująca się w , zamieniana jest na format \texttt{yyyy-MM-dd}, czyli rok-miesiąc-dzień.

        Dodatkowo, jeżeli tag jest \dicomtag{PatientAge}{0010}{1010} obecny wyświetlany jest także wiek pacjenta.

        \item TagStudyDescription
        \item TagSeriesDescription
    \end{itemize}

    \item \sokarclass{HospitalDataIndicator}
    \item \sokarclass{ImageOrientationIndicator}
    \item \sokarclass{ModalityIndicator}
    \item \sokarclass{PixelSpacingIndicator}
\end{itemize}

\paragraph{Podział obowiązków w klasie}
\sokarclass{DicomScene} dzieli obowiązki wyświetlania poszczególnych rzeczy innym obiektom, które dziedziczą po \sokarclass

\paragraph{Interfejs generowanie obrazu}
\sokarclass{DicomScene} to klasa abstrakcyjna i nie implementuje sposoby generowania obrazu, ale implementuje interfejs do generowania obrazu, który wykorzystują klasy dziedziczące po \sokarclass{DicomScene}.
Sposoby generowania obrazu są opisane w sekcji ... ;

\paragraph{Informacje o obrazie}
\sokarclass{DicomScene} jest odpowiedzialny za wyświetlenie informacji o obrazie, które nie są zależne od interpretacji woksela obrazu.
Czyli takie jak wielkość obrazka w skali rzeczywistej zapisana w \dicomtag{PixelSpacing}{0028}{0030} określający wielkość woksela w milimetrach.
Szczegóły wyświetlania informacji są zawarte w sekcji ... ;

Funkcjonalność(zajżenć do UML)

Współbierzność

Struktura danych

Dlaczego wektor, a nie drzewo

\subsection{Generowanie obrazów z danych}

\subsubsection{Monochorme}

Obrazy

